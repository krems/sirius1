\documentclass[russian,aspectratio=169,14pt]{beamer}

\usetheme{ShipilevRH}

\title{ООП. SOLID}
\subtitle{Пишем поддерживаемый код}
\author{Валерий Алексеевич Овчинников}
\institute{valery.ovchinnikov@phystech.edu}

\begin{document}

\maketitle

\section{Мотивация}

\begin{frame}
	\frametitle{Чего мы хотим от кода?}
	\begin{enumerate}
		\item Легко читать
		\item Легко поддерживать
		\item Легко расширять
		\pause
		\item High cohesion and loose coupling\\
		Сильная связность при слабой связанности (запутанности)
	\end{enumerate}
\end{frame}

\begin{frame}
	\frametitle{Зачем коду быть понятным?}
	\begin{enumerate}
		\item Код пишут один раз, а читают много
		\pause
		\item Не нужно читать документацию, чтобы разобраться
		\pause
		\item Проще искать ошибки
		\pause
		\item Проще менять и дополнять
	\end{enumerate}
\end{frame}

\begin{frame}[fragile]
	\frametitle{Зачем коду быть понятным?}
	Что здесь происходит?
	\vfill
	\begin{listjava}
double f(int a, int b) {
	double y = b;
	double z = 1;
	while (a > 0) {
		if (a % 2 == 1)
			z *= y;
		y = Math.pow(y, 2);
		a /= 2;
	}
	return z;
}
	\end{listjava}
\end{frame}

\begin{frame}
	\frametitle{Зачем коду быть понятным?}
	\begin{center}
	\includegraphics[height=0.7\textheight]{wut.jpg}
	\end{center}
\end{frame}

\begin{frame}[fragile]
	\frametitle{Зачем коду быть понятным?}
	Что здесь происходит?
	\vfill
	\begin{listjava}
double fastPow(double base, int power) {
    double result = 1;
    while (power > 0) {
        if (isOdd(power)) {
            result *= base;
        }
        base = Math.pow(base, 2);
        power /= 2;
    }
    return result;
}
	\end{listjava}
\end{frame}

\section{ООП}

\begin{frame}
	\frametitle{Записка на полях}
	Интерфейс -- набор методов, через которые внешний мир может контактировать с объектом
\end{frame}

\begin{frame}
	\begin{center}
	\includegraphics[height=0.8\textheight]{3_words.jpeg}
	\end{center}
\end{frame}

\begin{frame}
	\frametitle{Основные идеи}
	\begin{enumerate}
		\item Инкапсуляция
		\item Наследование (типизация)
		\item Полиморфизм
		\item Абстракция
	\end{enumerate}
\end{frame}

\begin{frame}
	\frametitle{Инкапсуляция}
	\begin{itemize}
		\item Сокрытие внутреннего устройства объекта
		\item Контроль над изменением состояния объекта
	\end{itemize}
\end{frame}

\begin{frame}
	\frametitle{Наследование}
	\begin{itemize}
		\item Инструмент для создания иерархии типов, достижения полиморфизма
		\item Позволяет переиспользовать код
	\end{itemize}
\end{frame}

\begin{frame}
	\frametitle{Полиморфизм}
	\begin{itemize}
		\item Позволяет работать с объектами разного типа через общий интерфейс
	\end{itemize}
\end{frame}

\begin{frame}
	\frametitle{Абстракция}
	Чтобы всё вышеперечисленное работало, нужно правильно выбирать абстракции\\
	Оставлять в интерфейсе только значимые общие элементы для всех объектов, реализующих этот интерфейс
\end{frame}

\section{SOLID}

\begin{frame}
	\frametitle{Расшифровка}
	\begin{enumerate}
		\item Single Responsibility Principle (SRP)\\
		Принцип единственной обязанности (цели)
		\item Open/Closed Principle (OCP)\\
		Принцип открытости/закрытости интерфейсов
		\item Liskov Substituition Principle (LSP)\\
		Принцип подстановки Лисков
		\item Interface Segregation Principle (ISP)\\
		Принцип разделения интерфейсов
		\item Dependency Inversion Principle (DIP)\\
		Принцип инверсии зависимостей
	\end{enumerate}
\end{frame}

\begin{frame}
	\frametitle{Принцип единственной обязанности (цели)}
	\begin{minipage}{1.0\textwidth}
	Каждый класс должен представлять одну сущность. Должна быть только одна причина, чтобы \textbf{менять} этот класс\\
	Каждый метод должен делать одно какое-то действие\\
	\end{minipage}
	\begin{minipage}{1.0\textwidth}
	Позволяет легко ориентироваться в коде и быстро понимать в каком компоненте ошибка\\
	Легко заменить реализацию конкретного функционала
	\end{minipage}
\end{frame}

\begin{frame}
	\frametitle{Принцип открытости/закрытости интерфейсов}
	\begin{minipage}{1.0\textwidth}
	Интерфейсы должны быть открыты для расширения, но закрыты для изменения\\
	\end{minipage}
	\begin{minipage}{1.0\textwidth}
	Интерфейс это контракт, если меняется интерфейс, то ломается код, использующий его\\
	Добавление новых методов в интерфейс не ломает существующий код, но расширяет возможности
	\end{minipage}
\end{frame}

\begin{frame}
	\frametitle{Принцип подстановки Лисков}
	\begin{minipage}{1.0\textwidth}
	Если вместо объектов базового типа подставить объекты его подтипа, то код должен продолжать корректно работать\\
	\end{minipage}
	\begin{minipage}{1.0\textwidth}
	Позволяет менять реализации, сохраняя интерфейс и не меняя остальной код
	\end{minipage}
\end{frame}

\begin{frame}
	\frametitle{Принцип разделения интерфейсов}
	\begin{minipage}{1.0\textwidth}
	Лучше иметь много маленьких, специализированных интерфейсов, чем один всемогущий\\
	Перекликается с SRP\\
	\end{minipage}
	\begin{minipage}{1.0\textwidth}
	Помогает реализовать предыдущие принципы
	\end{minipage}
\end{frame}

\begin{frame}
	\frametitle{Принцип инверсии зависимостей}
	\begin{minipage}{1.0\textwidth}
	Абстракции не должны зависеть от деталей, детали должны зависеть от абстракций\\
	Иными словами на заданном уровне должен использоваться один и тот же уровень абстракции\\
	\end{minipage}
	\begin{minipage}{1.0\textwidth}
	Помогает лучше выбирать абстракцию, сохранять инкапсуляцию, улучшает читаемость кода
	\end{minipage}
\end{frame}

\section{Другие принципы}

\begin{frame}
	\frametitle{Tell, don't ask}
	\begin{minipage}{1.0\textwidth}
	Говорите объекту что нужно сделать, а не спрашивайте о его состоянии\\
	Состояние класса не должно раскрываться даже через методы (например get/set)\\
	Объект должен сам основываясь на своем состоянии совершать действия\\
	\end{minipage}
	\begin{minipage}{1.0\textwidth}
	Обеспечивает логическую инкапсуляцию
	\end{minipage}
\end{frame}

\begin{frame}
	\frametitle{Предпочитайте агрегацию наследованию}
	\begin{minipage}{1.0\textwidth}
	Лучше реализовать интерфейс расширяемого класса и завести в новом классе поле типа расширяемого класса, при необходимости вызывать нужные методы расширяемого класса\\
	\end{minipage}
	\begin{minipage}{1.0\textwidth}
	Позволяет избегать глупых и неочевидных ошибок, строитб более правильную с точки зрения ООП иерархию типов
	\end{minipage}
\end{frame}



\questions

\end{document}
